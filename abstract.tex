\begin{abstractzh}
這是中文行距測試,應該看到一點五倍行距。這是中文行距測試,應該看到一點
五倍行距。這是中文行距測試,應該看到一點五倍行距。這是中文行距測試,應
該看到一點五倍行距。這是中文行距測試,應該看到一點五倍行距。這是中文行
距測試,應該看到一點五倍行距。這是中文行距測試,應該看到一點五倍行距。
這是中文行距測試,應該看到一點五倍行距。這是中文行距測試,應該看到一點
五倍行距。這是中文行距測試,應該看到一點五倍行距。這是中文行距測試,應
該看到一點五倍行距。這是中文行距測試,應該看到一點五倍行距。這是中文行
距測試,應該看到一點五倍行距。這是中文行距測試,應該看到一點五倍行距。 \\

\noindent
關鍵字:台大、公館、羅斯福路、德田館
\end{abstractzh}

\begin{abstracten}
In computer vision, image classification always have an important place due to its widespread applications such as image compression, object tracking, image analysis and ground use evolution. In this particular area we will use classification to extract road networks from satellites images. It consists of giving a label to each pixel if it belongs or not to a road, the applications are very wide from Network mapping to monitor territories.

To extract the roads accurately we propose a machine learning based technique with additional classification algorithm as post-processing. Our procedure is described as follows: first the input image is cutted in small squares, it will fit better to a reduced neural network and will decrease computation time and provide more detailed information, we also increase the range of input data using color’s pre-processing algorithm. Then we exploit color channels and additional gradient channels to put inside the neural network that is trained to classify road and non-road pixels. Finally we verify with additional classification algorithm the detected roads and with post-processing complete the missing road parts to construct the final road map.

Simulations show that our proposed method detect most of the road pixels and outperforms state-of-the-art methods.\\

\noindent
\textbf{Index terms}: Satellite Images, Image classification, Computer Vision, Road Extraction, Machine Learning, Neural Network
\end{abstracten}
